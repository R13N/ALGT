\section{Tutorial: developing a simple programming
language}\label{tutorial-developing-a-simple-programming-language}

We will develop a programming language which can work with booleans and
integers. Apart from doing basic arithmetic, we can apply anonymous
functions on them.

Example programs, and the value they become, are

\begin{longtable}[c]{@{}ll@{}}
\caption{Example programs and their outcome}\tabularnewline
\toprule
Program & Evaluates to\tabularnewline
\midrule
\endfirsthead
\toprule
Program & Evaluates to\tabularnewline
\midrule
\endhead
\texttt{True} & \texttt{True}\tabularnewline
\texttt{1} & \texttt{1}\tabularnewline
\texttt{24\ +\ 18} & \texttt{42}\tabularnewline
`If True Then 1 Else & 2\texttt{`1}\tabularnewline
\texttt{1\ +\ (2\ +\ 3)} & \texttt{6}\tabularnewline
`(\textbackslash{} x : Int . x + 1 & ) 41\texttt{`42}\tabularnewline
\bottomrule
\end{longtable}

The expression \texttt{(\textbackslash{}x\ :\ Int\ .\ x\ +\ 1)} is a
\emph{lambda expression}. This is an anonymous function, taking one
argument - named x- of type \texttt{Int}. When applied (e.g.
\texttt{(\textbackslash{}x\ :\ Int\ .\ x\ +\ 1)\ 41}, the expression
right of the \texttt{.} is returned, with the variable \texttt{x}
substituted by the argument, becoming \texttt{41\ +\ 1}.

\subsection{Setting up a .language}\label{setting-up-a-.language}

A language is declared inside a \texttt{.language} file \footnote{Actually,
  the extension doesn't matter at all.}. Create \texttt{STFL.language},
and put a title in it:

`

\subsection{Declaring the syntax}\label{declaring-the-syntax}

For a full reference on syntax, see the
\protect\hyperlink{syntax}{reference manual on syntax}

A program is nothing more then a string of a specific form. To describe
strings of an arbitrary structure, \emph{BNF} \footnote{Backus-Naur-form,
  as introduced by John Backus in the ALGOL60-report. TODO reference}
can be used.

Consider all possible boolean forms: \texttt{True} and \texttt{False}.

\subsection{Functions}\label{functions}

\subsection{Relations and Rules}\label{relations-and-rules}

\subsection{Properties}\label{properties}

\subsection{Recap: used command line
arguments}\label{recap-used-command-line-arguments}
